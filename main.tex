% resume.tex
%
% (c) 2002 Matthew Boedicker <mboedick@mboedick.org> (original author) http://mboedick.org
% (c) 2003-2007 David J. Grant <davidgrant-at-gmail.com> http://www.davidgrant.ca (fork 1)
% (c) 2016 Ananda Seelan <ananda.seelan@gmail.com> (fork 2)
% (c) 2024 Ariharasudhan <aravindariharan@gmail.com> https://arihara-sudhan.github.io (fork 3)
% This work is licensed under the Creative Commons Attribution-ShareAlike 3.0 Unported License. To view a copy of this license, visit http://creativecommons.org/licenses/by-sa/3.0/ or send a letter to Creative Commons, 171 Second Street, Suite 300, San Francisco, California, 94105, USA.

\documentclass[letterpaper,11pt]{article}

%-----------------------------------------------------------
%Margin setup

\setlength{\voffset}{0.1in}
\setlength{\paperwidth}{8.5in}
\setlength{\paperheight}{11in}
\setlength{\headheight}{0in}
\setlength{\headsep}{0in}
\setlength{\textheight}{11in}
\setlength{\textheight}{9.5in}
\setlength{\topmargin}{-0.25in}
\setlength{\textwidth}{7in}
\setlength{\topskip}{0in}
\setlength{\oddsidemargin}{-0.25in}
\setlength{\evensidemargin}{-0.25in}

\usepackage{enumitem}
\usepackage{fancyhdr}
\usepackage{lmodern}
%-----------------------------------------------------------
%\usepackage{fullpage}
%\textheight=9.0in
\pagestyle{empty}
\raggedbottom
\raggedright
\setlength{\tabcolsep}{0in}

%-----------------------------------------------------------
%Custom commands
\newcommand{\resitem}[1]{\item #1 \vspace{-2pt}}
\newcommand{\resheading}[1]{\textbf{\sffamily{\mbox{~}{\large #1} \vphantom{p\^{E}}}}}
%{{\large \textbf{#1 \vphantom{p\^{E}}}}}
\newcommand{\ressubheading}[4]{
\begin{tabular*}{6.5in}{l@{\extracolsep{\fill}}r}
		\textbf{#1} & #2 \\
		\textit{#3} & \textit{#4} \\
\end{tabular*}\vspace{-6pt}}
\newcommand\blfootnote[1]{%
  \begingroup
  \renewcommand\thefootnote{}\footnote{#1}%
  \addtocounter{footnote}{-1}%
  \endgroup
}
%-----------------------------------------------------------
\pagestyle{fancy}
\fancyhead{} % clear all header fields
\renewcommand{\headrulewidth}{0pt} % no line in header area
\fancyfoot{} % clear all footer fields
\fancyfoot[LE,RO]{\thepage}           % page number in "outer" position of footer line
\fancyfoot[RE,LO]{\fontsize{7}{4}\selectfont Ariharasudhan - Resume}
\begin{document}

\begin{tabular*}{7in}{l@{\extracolsep{\fill}}r}
\textbf{\Large Ariharasudhan}  & 91 6382509390\\
https://arihara-sudhan.github.io &  aravindariharan@gmail.com \\
\end{tabular*}
\\
\vspace{0.05in}
\noindent\makebox[\linewidth]{\rule{\linewidth}{0.4pt}}
\vspace{0.025in}

\resheading{Education}
\begin{itemize}
\item[]
	\ressubheading{Einstein College of Engineering}{Tirunelveli, India}{Bachelor of Engineering, Computer Science and Engineering (CGPA: 8.83)}{2019 - 2023}

\item[]
	\ressubheading{Government Hr. Sec. School}{Tenkasi, India}{Tamil Nadu Higher Secondary School Examination (Score: 87\%)}{2019}

\end{itemize}

\resheading{Programming Skills}
\vspace{-2mm}
\begin{itemize}
\item[]{\textit{Deep Learning Frameworks:} PyTorch, NumPy}\vspace{-2mm}
\item[]{\textit{Languages:} Python, Java, JavaScript, C}\vspace{-2mm}
\item[]{\textit{Databases:} MySQL, MongoDB}\vspace{-2mm}
\item[]{\textit{Web Development Frameworks:} FastAPI, ReactJS, NodeJS, Express JS}\vspace{-2mm}
\item[]{\textit{Operating Systems:} Windows, MacOS, Linux (mainly Ubuntu)}\vspace{-2mm}
\item[]{\textit{Version Control Systems:} Git}\vspace{-2mm}
\end{itemize}

\vspace{0.05in}
\resheading{Work Experience}
\begin{itemize}
\item[]
	\ressubheading{Zoho Corporation}{Tenkasi, India}{Member Technical Staff, Zoho Desk}{April 2024 - Now}
	\begin{itemize}
		\resitem{Working on The UI & AI Currently}
		\resitem{\textit{I am currently developing a Retrieval-Augmented Generation (RAG) system leveraging the Zoho Desk Knowledge Base as a source of information. The system integrates FAISS (Facebook AI Similarity Search) for efficient indexing and retrieval of knowledge base documents. Additionally, it employs a HuggingFace large language model (LLM) to enhance the generation of contextually accurate and insightful responses. I also write production-level JavaScript and ReactJS code to implement new features.}}
	\end{itemize}

\item[]
	\ressubheading{Zoho Corporation}{Trichy, India}{Member Technical Staff, ZLabs Speech Profiling}{Jan 2024 - April 2024}
	\begin{itemize}
		\resitem{Learned about Audio Feature Extraction}
		\resitem{\textit{I conducted audio classification using CNNs, achieving peak accuracy. I addressed challenges with varying sampling rates and gained valuable insights into audio engineering and feature extraction.}}
	\end{itemize}
	
\item[]
	\ressubheading{Zoho Corporation}{Tripur, India}{Member Technical Staff, ZLabs Intelligent Document Processing}{Jun 2023 - Dec 2023}
	\begin{itemize}
		\resitem{Developed A Fewshot Doc2Vec Model}
		\resitem{\textit{I began my research with ResNet-50 and then transitioned to DiT, a vision transformer pretrained on a document dataset. We extracted document embeddings and performed few-shot training using Triplet Loss. With well-labeled data, we achieved a similarity search accuracy of 100\%. Later, in my research, I discovered interesting insights, such as classification models sometimes performing like few-shot models and vice versa. Finally, I conducted multimodal training by combining visual features from ResNet and textual features from BERT, resulting in strong generalization.}}
	\end{itemize}

\item[]
	\ressubheading{Zoho Corporation}{Coimbatore, India}{Project Trainee, ZLabs Intelligent Document Processing}{Jan 2023 - Jun 2023}
	\begin{itemize}
		\resitem{Learned Deep Learning Computer Hardwares, Backend Frameworks and Revised Python}
		\resitem{\textit{I meticulously learned deep learning, which sparked my interest in creating small books (https://arihara-sudhan.github.io/books). I successfully trained small neural networks and tackled a few-shot classification task, which I resolved using contrastive loss.}}
	\end{itemize}

\end{itemize}

\resheading{Projects}
\begin{itemize}
\item[]
	\textbf{AI Powered MediKit}
	\vspace{-3mm}
	\begin{itemize}
		\resitem{The AI-Powered MediKit enhances medical analysis with advanced features, utilizing Vision Transformers (ViTs) for improved classification over conventional CNNs. ViTs excel in capturing subtle details in medical images that CNNs struggle with. It also offers Heartbeat Analysis using MFCC for identifying abnormalities. The MediKit supports few-shot classification for tablets, reducing the need for retraining with new data. Additionally, it includes a feature for users to inquire about herbal remedies.}
	\end{itemize}

\item[]
	\textbf{Grouped Detection of Objects}
	\vspace{-3mm}
	\begin{itemize}
		\resitem{The Grouped Detection of Objects utilizes the YOLO algorithm to perform accurate object detection in images. After detecting the objects, I crop specific regions of interest and feed them into a Swin Transformer-backboned few-shot network, which helps in performing efficient few-shot classification. To determine the optimal number of clusters, I incorporate the Elbow Method along with K-means clustering, ensuring the model is well-tuned. For seamless deployment, I use FastAPI to create a robust microservice that handles real-time predictions and can be easily scaled for various use cases.}
	\end{itemize}

\item[]
	\textbf{Fewshot Classify Anything Model}
	\vspace{-3mm}
	\begin{itemize}
	    \resitem{The "Classify Anything" model performs similarity searches on stored embeddings of Images. We use a Swin Transformer as the backbone and train the embeddings using triplet loss. With a few extracted embeddings saved in an index, the model classifies all inputs effectively.}
	\end{itemize}

\item[]
	\textbf{Thirukkural - AI Similarity Search}
	\vspace{-3mm}
	\begin{itemize}
		\resitem{The Thirukkural AI Similarity Search utilizes paraphrase transformer embeddings to perform searches based on queries entered by users. It is served using FastAPI, with the frontend currently implemented in ReactJS.}
	\end{itemize}

\end{itemize}

\resheading{Achievements/Activities}
\vspace{-3mm}
\begin{itemize}
    \item Assisting a startup called Yash in automating 3D texture application using Python and training simple ML models
    \vspace{-3mm}
    \item Taught MERN stack development to a group of expert students from my college, and ML to a group of expert students from colleges in Coimbatore, Tirunelveli, and Madurai
    \vspace{-3mm}
    \item Qualified for the final round of Medecro.ai's Hackathon - Built an AI-powered MediKit
    \vspace{-3mm}
    \item Qualified for the final round of the Atheneum Hackathon, conducted by IGDTUW, New Delhi - Built a Smart Education System
    \vspace{-3mm}
    \item Secured I positions in Hackathons, Code Debugging, Web Development, and other competitions conducted by colleges in the south, such as Rohini College of Engineering, Thamirabarani Engineering College, and PSN College of Engineering
    \vspace{-3mm}
    \item Awarded Best Project Presentation in the Project Expo at Einstein College of Engineering for the Smart Attendance System
    \vspace{-3mm}
    \item Secured first position for showcasing skills in Mathematics in the Talent Search Examination conducted by JP College of Engineering, Tenkasi
    \vspace{-3mm}
    \item Secured the Yuva Shri Kala Bharathi Award from Bharathi Yuva Kendra, Madurai - For outstanding performance in education and the arts
\end{itemize}
\vspace{2cm}
\end{document}
