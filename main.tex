% resume.tex
%
% (c) 2002 Matthew Boedicker <mboedick@mboedick.org> (original author) http://mboedick.org
% (c) 2003-2007 David J. Grant <davidgrant-at-gmail.com> http://www.davidgrant.ca (fork 1)
% (c) 2016 Ananda Seelan <ananda.seelan@gmail.com> (fork 2)
% (c) 2024 Ariharasudhan <aravindariharan@gmail.com> https://arihara-sudhan.github.io (fork 3)
% Licensed under the Creative Commons Attribution-ShareAlike 3.0 Unported License.

\documentclass[letterpaper,11pt]{article}
\usepackage[utf8]{inputenc}
\usepackage[T1]{fontenc}

%-----------------------------------------------------------
% Margin setup
\setlength{\voffset}{0.1in}
\setlength{\paperwidth}{8.5in}
\setlength{\paperheight}{11in}
\setlength{\headheight}{0in}
\setlength{\headsep}{0in}
\setlength{\textheight}{9.5in} % Removed redundant definition
\setlength{\topmargin}{-0.25in}
\setlength{\textwidth}{7in}
\setlength{\topskip}{0in}
\setlength{\oddsidemargin}{-0.25in}
\setlength{\evensidemargin}{-0.25in}

% Packages
\usepackage{enumitem}
\usepackage{fancyhdr}
\usepackage{lmodern}
\usepackage{url} % Added for better URL formatting

\pagestyle{empty}
\raggedbottom
\raggedright
\setlength{\tabcolsep}{0in}

%-----------------------------------------------------------
% Custom commands
\newcommand{\resitem}[1]{\item #1 \vspace{-2pt}}
\newcommand{\resheading}[1]{\textbf{\sffamily{\mbox{~}{\large #1} \vphantom{p\^{E}}}}}
\newcommand{\ressubheading}[4]{
\begin{tabular*}{6.5in}{l@{\extracolsep{\fill}}r}
    \textbf{#1} & #2 \\
    \textit{#3} & \textit{#4} \\
\end{tabular*}\vspace{-4pt}} % Reduced spacing from -6pt to -4pt for better alignment

\newcommand\blfootnote[1]{%
  \begingroup
  \renewcommand\thefootnote{}\footnote{#1}%
  \addtocounter{footnote}{-1}%
  \endgroup
}

%-----------------------------------------------------------
% Fancy footer settings
\pagestyle{fancy}
\fancyhead{} % Clear all header fields
\renewcommand{\headrulewidth}{0pt} % No line in header area
\fancyfoot{} % Clear all footer fields
\fancyfoot[LE,RO]{\thepage} % Page number in "outer" position of footer
\fancyfoot[RE,LO]{\fontsize{9}{6}\selectfont Ariharasudhan - Resume} % Increased footer text size

%-----------------------------------------------------------
\begin{document}

% Contact Information
\begin{tabular*}{7in}{l@{\extracolsep{\fill}}r}
\textbf{\Large Ariharasudhan}  & +91 6382509390\\
\url{https://arihara-sudhan.github.io} &  aravindariharan@gmail.com \\ % Used \url{} for better line-breaking
\end{tabular*}
\\
\vspace{0.05in}
\noindent\makebox[\linewidth]{\rule{\linewidth}{0.4pt}}
\vspace{0.025in}

%-----------------------------------------------------------
% Education
\resheading{Education}
\begin{itemize}
\item[]
    \ressubheading{Einstein College of Engineering}{Tirunelveli, India}{B.E., Computer Science and Engineering (CGPA: 8.83)}{2019 - 2023}

\item[]
    \ressubheading{Government Hr. Sec. School}{Tenkasi, India}{Higher Secondary School Examination (Score: 87\%)}{2019}
\end{itemize}

%-----------------------------------------------------------
% Programming Skills
\resheading{Programming Skills}
\vspace{-2mm}
\begin{itemize}
\item[]{\textit{Deep Learning Frameworks:} PyTorch, NumPy}\vspace{-2mm}
\item[]{\textit{Languages:} Python, Java, JavaScript, C}\vspace{-2mm}
\item[]{\textit{Databases:} MySQL, MongoDB}\vspace{-2mm}
\item[]{\textit{Web Development Frameworks:} FastAPI, ReactJS, NodeJS, Express.js}\vspace{-2mm}
\item[]{\textit{Operating Systems:} Windows, macOS, Linux (mainly Ubuntu)}\vspace{-2mm}
\item[]{\textit{Version Control Systems:} Git}\vspace{-2mm}
\end{itemize}

%-----------------------------------------------------------
% Work Experience
\vspace{2mm}
\resheading{Work Experience}
\begin{itemize}
\item[]
    \ressubheading{Moative}{Chennai, India}{AI/ML Engineer}{June 2025 - Present}
    \begin{itemize}
        \resitem{Developing Full Stack AI Systems, encompassing everything from model prototyping to scalable cloud deployment}
        \resitem{\textit{I am developing an AI-powered system to detect account overlaps between companies using partner graphs. When overlaps are identified, the system classifies the relationship (e.g., Tier 1, strong enablement) and categorizes the opportunity (e.g., Prospect-Customer). Being built with Python and FastAPI, the pipeline integrates partner data streams, applies business rule-based reasoning, and alerts sales teams via escalation workflows. I also designed a fallback automation that sends personalized emails to key contacts when no sales rep responds—enabling competitive intelligence, cross-sell actions, and accelerating deals. The system converts latent overlap into active revenue opportunities.}}
	\resitem{\textit{I developed a web-based AI platform that automates CEQA permitting by integrating OCR, advanced NLP, and vector-based document analysis. Leveraging Google's Gemini API, the system classifies uploaded PDFs into CEQA categories (Exempt, ND, MND, EIR), generates justification reports, and flags environmental concerns. I built a FastAPI backend with Gemini-powered classification, real-time caching, and PDF generation, alongside an interactive frontend with traffic-light status indicators, map-based location analysis, and a chatbot assistant. The platform significantly accelerates environmental compliance review for consultants, agencies, and developers.}}
	\resitem{\textit{I engineered a graph-based clinical trial analysis platform focused on Antibody-Drug Conjugates (ADC), integrating Neo4j, FastAPI, and Google Gemini AI to enable natural language querying over pharmacokinetic and safety datasets. The system dynamically generates Cypher queries from plain-English questions, visualizes clinical parameters (AUC, Cmax, AE grades) via interactive dashboards, and provides AI-generated insights. I deployed it into AWS.}}
    \end{itemize}
\newpage
\item[]
    \ressubheading{Zoho Corporation}{Tenkasi, India}{Member Technical Staff, Zoho Desk}{April 2024 - May 2025}
    \begin{itemize}
        \resitem{Developing UI and AI systems for Zoho Desk.}
        \resitem{\textit{Worked on a Retrieval-Augmented Generation (RAG) system using FAISS for document retrieval and a HuggingFace LLM for improved responses. Also developing production-grade JavaScript and ReactJS features.}}
    \end{itemize}
\item[]
	\ressubheading{Zoho Corporation}{Trichy, India}{Member Technical Staff, ZLabs Speech Profiling}{Jan 2024 - April 2024}
	\begin{itemize}
		\resitem{Learned about Audio Feature Extraction}
		\resitem{\textit{I conducted audio classification using CNNs, achieving peak accuracy. I addressed challenges with varying sampling rates and gained valuable insights into audio engineering and feature extraction.}}
	\end{itemize}
	
\item[]
	\ressubheading{Zoho Corporation}{Tripur, India}{Member Technical Staff, ZLabs Intelligent Document Processing}{Jun 2023 - Dec 2023}
	\begin{itemize}
		\resitem{Developed A Fewshot Doc2Vec Model}
		\resitem{\textit{I began my research with ResNet-50 and then transitioned to DiT, a vision transformer pretrained on a document dataset. We extracted document embeddings and performed few-shot training using Triplet Loss. With well-labeled data, we achieved a similarity search accuracy of 100\%. Later, in my research, I discovered interesting insights, such as classification models sometimes performing like few-shot models and vice versa. Finally, I conducted multimodal training by combining visual features from ResNet and textual features from BERT, resulting in strong generalization.}}
	\end{itemize}

\item[]
	\ressubheading{Zoho Corporation}{Coimbatore, India}{Project Trainee, ZLabs Intelligent Document Processing}{Jan 2023 - Jun 2023}
	\begin{itemize}
		\resitem{Learned Deep Learning Computer Hardwares, Backend Frameworks and Revised Python}
		\resitem{\textit{I meticulously learned deep learning, which sparked my interest in creating small books (https://arihara-sudhan.github.io/books). I successfully trained small neural networks and tackled a few-shot classification task, which I resolved using contrastive loss.}}
	\end{itemize}

\end{itemize}

%-----------------------------------------------------------
% Projects
\resheading{Projects}
\begin{itemize}

\item[]
	\textbf{AI Powered MediKit}
	\vspace{-3mm}
	\begin{itemize}
		\resitem{Revolutionizes healthcare diagnostics by leveraging Vision Transformers (ViTs) for precise medical image analysis.}
		\resitem{Overcomes CNN limitations by capturing fine-grained medical features with superior accuracy.}
		\resitem{Features Heartbeat Analysis using Mel-Frequency Cepstral Coefficients (MFCC) to classify heartbeats and detect abnormalities.}
		\resitem{Supports few-shot classification for tablet identification, minimizing retraining needs and improving adaptability.}
		\resitem{Includes a Herbal Solution feature that bridges modern AI with traditional medicine by suggesting natural remedies.}
		\resitem{Combines precision, innovation, and accessibility to set a new benchmark in AI-driven medical solutions.}
	\end{itemize}
\newpage
\item[]
	\textbf{MindKural - RAG System}
	\vspace{-3mm}
	\begin{itemize}
		\resitem{Developed a conversational AI bot using LangChain to answer queries based on Thirukkural, the timeless Tamil literary masterpiece.}
		\resitem{Utilized FAISS for vector embeddings and similarity search to enable efficient retrieval.}
		\resitem{Enhanced responses using Retrieval-Augmented Generation (RAG) with Falcon LLM.}
		\resitem{Transformed Thirukkural’s wisdom into a knowledge framework that informs and heals.}
	\end{itemize}

\item[]
	\textbf{Grouped Detection of Objects}
	\vspace{-3mm}
	\begin{itemize}
		\resitem{Utilized the YOLO algorithm to detect objects in images accurately.}
		\resitem{Cropped regions of interest and applied a Swin Transformer-based few-shot network for classification.}
		\resitem{Implemented K-means clustering with the Elbow Method to optimize object grouping.}
		\resitem{Deployed a scalable real-time prediction microservice using FastAPI.}
	\end{itemize}

\item[]
	\textbf{Fewshot Classify Anything Model}
	\vspace{-3mm}
	\begin{itemize}
	    \resitem{Performs similarity searches on stored embeddings of images.}
	    \resitem{Uses a Swin Transformer as the backbone and employs triplet loss for training embeddings.}
	    \resitem{Efficiently classifies new inputs by referencing a few saved embeddings in an index.}
	\end{itemize}

\item[]
	\textbf{Next-Word Prediction using Bigram Model}
	\vspace{-3mm}
	\begin{itemize}
	    \resitem{Implemented a probabilistic language model that predicts the next word based on the previous word using a bigram approach.}
	    \resitem{Utilized joint probability and conditional probability to compute word sequences.}
	    \resitem{Demonstrated real-world text generation applications by starting with seed words and generating coherent sentences.}
	\end{itemize}

\item[]
	\textbf{Dialect Classification using Naïve Bayes}
	\vspace{-3mm}
	\begin{itemize}
	    \resitem{Built a text classification model for dialect detection using the Naive Bayes algorithm.}
	    \resitem{Employed the Bag of Words approach to store word frequencies for classification.}
	    \resitem{Used Laplace Smoothing to handle zero probabilities and prevent model breakdown.}
	    \resitem{Optimized for efficiency by working in log-space to prevent underflow when multiplying small probabilities.}
	\end{itemize}

\end{itemize}

%-----------------------------------------------------------
% Achievements & Activities
\newpage
\resheading{Achievements/Activities}
\begin{itemize}
    \item Educating Peers at Zoho Desk on Client Technology  
          \textit{I mentor my team on JavaScript internals, the functional programming paradigm, ReactJS, and TypeScript.} (SEP 2024)
    \vspace{-2mm}
    
    \item Learning NLP and Speech Processing nightly at a study group meet (DAILY)
    \vspace{-2mm}

    \item Taught Basic Machine Learning to Students  
          \textit{Guided expert students from Coimbatore, Tirunelveli, and Madurai on Multi-Layer Perceptron and CNN via Google Meet.} (OCT 2024)
    \vspace{-2mm}

    \item Taught MERN Stack Development to Students  
          \textit{Conducted weekend sessions at Einstein College of Engineering on MERN Stack Development.} (DEC 2024)
    \vspace{-2mm}

    \item Session on "Meet AI" at Kamaraj College of Engineering and Technology, Virudhunagar  
          \textit{Taught AI fundamentals, fostering curiosity and learning to inspire an AI-powered South.} (JAN 2025)
    \vspace{-2mm}

    \item Session on "On Technology and Rural Development" at AKY Polytechnic College, Nellai  
          \textit{Discussed technology’s role in rural development, education, societal changes, and self-improvement.} (SEP 2024)
    \vspace{-2mm}

    \item Qualified for the final round of Medecro.ai's Hackathon  
          \textit{Developed AI-powered MediKit, a bundle of AI solutions for medical problems such as heartbeat analysis, tumor detection, and cell classification.} (AUG 2024)
    \vspace{-2mm}

    \item Qualified for the final round of the Atheneum Hackathon, IGDTUW, New Delhi  
          \textit{Proposed Smart Education System with OCR-based test evaluation, virtual pen, virtual quiz, and face recognition-based attendance.} (AUG 2022)
    \vspace{-2mm}

    \item Talent Search Examination, JP College of Engineering, Tenkasi  
          \textit{Secured 1st Prize for excellence in Mathematics and Science.} (FEB 2017)
    \vspace{-2mm}

    \item Yuva Shri Kala Bharathi Award, Bharathi Yuva Kenthra, Madurai  
          \textit{Recognized for outstanding performance in proficiency and arts.} (JAN 2018)
    \vspace{-2mm}

    \item Thirukkural Literature Explanations  
          \textit{Dedicating my evenings to writing detailed explanations for each Thirukkural.} (OCT 2024)
    \vspace{-2mm}

    \item Aladi Aruna Merit Scholarship, Einstein College of Engineering, Tirunelveli  
          \textit{Awarded a scholarship for securing a top rank in academics.} (FEB 2017)
    \vspace{-2mm}

    \item 1st Prizes in Paper Presentation, Web Designing, Code Debugging  
          \textit{Rohini College of Engineering, Kanyakumari - National Level Technical Symposium.} (MAR 2022)
    \vspace{-2mm}

    \item 1st Prizes in Paper Presentation and Web Designing Competitions  
          \textit{PSN College of Engineering, Tirunelveli - National Level Technical Symposium.} (APR 2022)
    \vspace{-2mm}

    \item 1st Prizes in Paper Presentation, Web Designing, and Technical Quiz  
          \textit{Thamirabarani Engineering College, Tirunelveli - National Level Technical Symposium.} (MAY 2022)
    \vspace{-2mm}

    \item Best Project Presentation in Mini Project Expo for III-year students  
          \textit{Developed an AI-Based Student Service System at Einstein College of Engineering.}
\end{itemize}
\vspace{2cm}
\end{document}
